
%%%%%%%%%%%%%%%%
% NEW CHAPTER! %
%%%%%%%%%%%%%%%%
\chapter{Second Chapter}

\begin{chapquote}{Author's name, \textit{Source of this quote}}
``This is a quote and I don't know who said this.''
\end{chapquote}

\section{Section heading}
\lipsum[1-2]

\subsection{Lorem ipsum dolor sit amet, consectetur adipiscing elit.}
\lipsum[1-2]

\begin{lstlisting}[language=XML]
<?xml version="1.0" encoding="utf-8"?>
<xs:schema attributeFormDefault="unqualified" 
    elementFormDefault="qualified" 
    xmlns:xs="http://www.w3.org/2001/XMLSchema">
  <xs:element name="points">
    <xs:complexType>
      <xs:sequence>
        <xs:element maxOccurs="unbounded" name="point">
          <xs:complexType>
            <!-- das ist ein test -->  
            <xs:attribute name="x" type="xs:unsignedShort" use="required" />
            <xs:attribute name="y" type="xs:unsignedShort" use="required" />
          </xs:complexType>
        </xs:element>
      </xs:sequence>
    </xs:complexType>
  </xs:element>
</xs:schema>
\end{lstlisting}

 
\begin{lstlisting}[language=Java, caption=Java Code example]
public class Test {

    public static final void main(String[] args){
        System.out.println("Hello World!");
    }
    
}
\end{lstlisting}



\begin{lstlisting}[language=XML, caption=XML Snippet example]
<?xml version="1.0" encoding="utf-8"?>
<xs:schema attributeFormDefault="unqualified" 
    elementFormDefault="qualified" 
    xmlns:xs="http://www.w3.org/2001/XMLSchema">
  <xs:element name="points">
    <xs:complexType>
      <xs:sequence>
        <xs:element maxOccurs="unbounded" name="point">
          <xs:complexType>
            <!-- das ist ein test -->  
            <xs:attribute name="x" type="xs:unsignedShort" use="required" />
            <xs:attribute name="y" type="xs:unsignedShort" use="required" />
          </xs:complexType>
        </xs:element>
      </xs:sequence>
    </xs:complexType>
  </xs:element>
</xs:schema>
\end{lstlisting}

\lipsum[1-2]

\subsection{Lorem ipsum dolor sit amet, consectetur adipiscing.}
\lipsum[1-2]

\subsection{Lorem ipsum dolor sit amet}
\lipsum[1-2]

\subsection{Lorem ipsum dolor sit amet, auctor et pulvinar non}
\lipsum[1-2] 


\section{Another section heading}
\lipsum[1-2]

%%%%%%%%%%%%%%%%%%%%%%%%%%%%%%%%%%%%%%%%%%%%%%%%%%%%%%%
% Sample table                                        %
% Source: www1.maths.leeds.ac.uk/latex/TableHelp1.pdf %
%%%%%%%%%%%%%%%%%%%%%%%%%%%%%%%%%%%%%%%%%%%%%%%%%%%%%%%
\begin{table}[ht]
\caption{Sample table} % title of Table
\centering % used for centering table
\begin{tabular}{c c c c}
% centered columns (4 columns)
\hline\hline %inserts double horizontal lines
S. No. & Column\#1 & Column\#2 & Column\#3 \\ [0.5ex]
% inserts table
%heading
\hline % inserts single horizontal line
1 & 50 & 837 & 970 \\
2 & 47 & 877 & 230 \\
3 & 31 & 25 & 415 \\
4 & 35 & 144 & 2356 \\
5 & 45 & 300 & 556 \\ [1ex] % [1ex] adds vertical space
\hline %inserts single line
\end{tabular}
\label{table:nonlin} % is used to refer this table in the text
\end{table}

Duis aute irure dolor in reprehenderit in voluptate velit esse cillum dolore eu fugiat nulla pariatur. Excepteur sint occaecat cupidatat non proident, sunt in culpa qui officia deserunt mollit anim id est laborum. \\ Lorem ipsum list:
\begin{itemize}
\item Mauris sit amet nulla mi, vitae rutrum ante.
\item Maecenas quis nulla risus, vel tincidunt ligula.
\item Nullam ac enim neque, non \emph{dapibus} mauris.
\end{itemize}
 

\noindent Lorem ipsum dolor sit amet, consectetur adipiscing elit. Duis risus ante, auctor et pulvinar non, posuere ac lacus. Praesent egestas nisi id metus rhoncus ac lobortis sem hendrerit. Etiam et sapien eget lectus interdum posuere sit amet ac urna\footnote{Lorem ipsum dolor sit amet, consectetur adipiscing elit. Duis risus ante, auctor et pulvinar non, posuere ac lacus.}:

\subsection{Lorem ipsum dolor sit amet, consectetur adipiscing elit.}
\lipsum[1-1]

{\hfill\includegraphics[width=0.4\textwidth]{test-img.pdf}\hfill}

\lipsum[1-1]

\subsection{Lorem ipsum dolor sit amet, consectetur adipiscing}
\lipsum[1-2]

%
% If pdf has more than 1 page
% 
\includepdf[pages={1}, nup=1x1, scale=0.8, pagecommand={}]{test-img.pdf}

%
% Include as graphics
%
\begin{figure}[ht]
	\centering
    \includegraphics[width=0.5\textwidth, angle=20]{test-img.pdf}
    \caption{This is some caption text}
\end{figure}
